%% start of file `moderncv_ntrp_template_en.tex'.
%% Copyright 2007 Xavier Danaux (xdanaux@gmail.com).
%
% This work may be distributed and/or modified under the
% conditions of the LaTeX Project Public License version 1.3c,
% available at http://www.latex-project.org/lppl/.
%
% Modded by ntrp (nitropowered@gmail.com)

\documentclass[10pt,a4paper]{moderncv}

\usepackage{xcolor}
\AfterPreamble{\hypersetup{
            colorlinks = true,
   		    linkcolor = blue,
            urlcolor  = blue,
            citecolor = blue,
            anchorcolor = blue
}}
% moderncv themes
%\moderncvtheme[blue]{casual}                 % optional argument are 'blue' (default), 'orange', 'red', 'green', 'grey' and 'roman' (for roman fonts, instead of sans serif fonts)
\moderncvtheme[blue]{classic}                % idem
\usepackage{amssymb}
\usepackage[T1]{fontenc}
% character encoding
\usepackage[utf8x]{inputenc}                   % replace by the encoding you are using
\usepackage[english]{babel}
\usepackage[scaled]{helvet}
\renewcommand\familydefault{\sfdefault} 
\usepackage[T1]{fontenc}
% adjust the page margins
\usepackage[scale=0.8]{geometry}
\recomputelengths                             % required when changes are made to page layout lengths

\fancyfoot{} % clear all footer fields
\fancyfoot[LE,RO]{\thepage}           % page number in "outer" position of footer line
\fancyfoot[RE,LO]{\footnotesize } % other info in "inner" position of footer line

% personal data
\firstname{Nicolò}
\familyname{Genesio}
%\title{Curriculum Vitae}               % optional, remove the line if not wanted
\address{Via Montevideo 8/12}{16129 Genova, Italy}    % optional, remove the line if not wanted
\mobile{3496437746}                    % optional, remove the line if not wanted
\email{nicogenesio91@gmail.com}                      % optional, remove the line if not wanted
%\extrainfo{additional information (optional)} % optional, remove the line if not wanted
\photo[84pt]{NicoloGenesio-3.jpg}                         % '64pt' is the height the picture must be resized to and 'picture' is the name of the picture file; optional, remove the line if not wanted
%\quote{"Two roads diverged in a wood, and I —
%I took the one less traveled by,
%And that has made all the difference" \newline -- Robert Frost}                 % optional, remove the line if not wanted

%\nopagenumbers{}                             % uncomment to suppress automatic page numbering for CVs longer than one page


%----------------------------------------------------------------------------------
%            content
%----------------------------------------------------------------------------------
\begin{document}
\maketitle

%Section
\section{Info}
\cvline{Birth}{\small 15/07/1991 (Finale Ligure)\normalsize}
\cvline{Citizenship}{\small Italian\normalsize}
\cvline{LinkedIn}{\small \url{https://www.linkedin.com/in/nicogene}\normalsize}
\cvline{Skype}{\small live:nicogene\normalsize}
\cvline{GitHub}{\small \url{https://github.com/Nicogene}\normalsize}
\cvline{Personal Site}{\small \url{https://nicogene.github.io/}\normalsize}

%Section
\section{Professional profile}
\cvline{}{\Large Robotics/Software engineer}
\cvline{}{\small My greatest interest lies in technology, especially the one applied to improve the life of human beings. 
During my work experience, my tasks were mainly focused on computer programming and software design applied to state-of-the-art robots.
Despite this, working in robotics allowed me to be contaminated by other disciplines like mechanical design, electronics, firmware development, control design. \newline
I possess good experience in computer vision, robotics, GUI design, communication protocols, embedded system development. 
My favorite programming language is C++, however, I have also familiarity with scripting languages such as Matlab, Python, and Lua.\newline
I experienced programming on all the main operating systems.
In my work experience, I acquired good teamwork skills and the ability to work on tasks out of my comfort zone.}
%Section
\section{Skills}
\subsection{Informatics} 
\cvline{Languages}{\textbf{Good knowledge:} C, C++17. \newline  \textbf{Intermediate knowledge:} Java, Matlab, CMake.  \newline
\textbf{Basic knowledge}: Lua, NI LabView, SQL, JSP, VB, HTML, Python.} 
\cvline{Platforms}{Linux, Windows, macOS.}
\cvline{Tools}{Qt creator, Creo Parametric, VS2010-19, gazebo, OpenCV, PCL, YARP, Git, Office suite, Matlab, Simulink, NI LabView, Swig, Embedded Wizard, Keil uvision.}
\subsection{Engineering}
\cvline{Topics}{\textbf{Good knowledge:} computer vision, robotics, software design. \newline 
\textbf{Intermediate knowledge:} model-based design, human-computer interaction.\newline
\textbf{Basic knowledge}: Electrotechnology, web application development, machine learning, mathematics, physics, biomedical instrumentation, chemistry, and biochemistry.}
\newpage
%Section
\section{Experience}
\subsection{Work}
\cventry{from 10/2016 to present}{Junior Engineer @ iCub Facility/iCub Tech}{Istituto Italiano di Tecnologia}{Genova}{Italy}
{
Currently, I am the Software Distro Manager, I am responsible for time-based software releases of the iCub-Tech department.
I participated in the development and the maintenance of \href{https://github.com/robotology/yarp}{YARP} robotics middleware, becoming one
of its top \href{https://github.com/robotology/yarp/graphs/contributors}{contributors}.
Moreover, I had the chance also of being involved in not-robotics projects like the \href{https://github.com/icub-tech-iit/ventilator-fi5}{ventilator fi5}, in particular for the GUI design of the embedded system.
During this experience, I acquired good familiarity with GUI design, communication protocols, and distributed systems. I have been also involved in a commercial project (protected by n.d.a) from which I learned how to deal with a large international company as a customer.
Finally, I participated also in dissemination activities such as \href{https://easy-peasy-robotics.github.io/editions/mfr17/course-program.html}{Easy Peasy Robotics 2017}-\href{https://easy-peasy-robotics.github.io/editions/web20/course-program.html}{2020}
and \href{http://www.icub.org/school/2018}{VVV school 2018}; I held the lesson with relative hands-on about YARP middleware.}
\cventry{from 02/2016 to 09/2016}{Support Engineer @ iCub Facility}{Istituto Italiano di Tecnologia}{Genova}{Italy}
{I worked on the humanoid robot iCub, in particular, I dealt with the design of the C++ module (\href{https://github.com/robotology-playground/Robust-View-Graph-SLAM}{Robust-View-Graph-SLAM}) that computes the 3D map of the scene using the images acquired from the eyes(cameras).}
%\subsection{Academic}
\cventry{from 04/2015 to 10/2015}{Internship @ iCub Facility}{Istituto Italiano di Tecnologia}{Genova}{Italy}{I worked on my Master thesis project supervised by Prof. Lorenzo Natale and Prof. Fabio Solari. My goal was to build a system able to classify the scene based on the traversability of each region using the 3D data acquired by the structured light depth sensor ASUS Xtion Pro Live.} % arguments 3 to 6 are optional
\cventry{from 03/2013 to 10/2013}{Internship @ PSPC lab}{Università Degli Studi di Genova}{Genova}{Italy}{I worked in the PSPC lab under the supervision of Prof. Fabio Solari for my bachelor thesis project. This project consisted on creating a \textit{gaze-tracker} C++ program, using OpenCV and a regular web camera.} % arguments 3 to 6 are optional

%Section
\section{Education}
\cventry{10/2018}{Modern C++}{KDAB}{Genova}{Italy}{I partecipated to a three-day course about modern c++ held by \href{https://www.kdab.com/}{KDAB}. This course talked about the novelties introducted by c++11-14-17.} %
\cventry{from 11/2015 to 01/2016}{Java Programming Course}{Azienda Sicura srl}{Milano}{Italy}{I participated to the Java Programming Course funded by \href{http://www.formatemp.it/it/}{Formatemp}. During this course I learnt how to build java and web applications. I experienced in using some tools as Eclipse J2EE, Apache Tomcat and Oracle DB.} 
\cventry{from 10/2013 to 10/2015}{Master Degree in Bioengineering}{Università degli studi di Genova}{Genova}{Italy}{Curriculum \textit{Neuroengineering and Bio-ICT}, 110/110 cum laude.} % arguments 3 to 6 are optional
\cventry{from 09/2010 to 10/2013}{Bachelor Degree in Biomedical Engineering}{Università degli studi di Genova}{Genova}{Italy}{108/110.} % arguments 3 to 6 are optional
\cventry{from 2006 to 2010}{Diploma Liceo Scientifico}{Liceo scientifico A.Issel}{Finale Ligure}{Italy}{100/100 cum laude.}

%Section
%\section{Master thesis}
%\cvline{title}{\emph{Title}}
%\cvline{supervisors}{Supervisors}
%\cvline{description}{\small Short thesis abstract}

%Section
\section{Languages}

\hspace{25mm}\small Self-assessment European level \href{http://europass.cedefop.europa.eu/en/resources/european-language-levels-cefr}{CEFR} (C2 maximum evaluation)\normalsize
\vspace{5mm}

\begin{tabular}{p{67mm} p{40mm} p{40mm} p{20mm}}
& \textbf{Understanding} & \textbf{Speaking} & \textbf{Writing} \\
\end{tabular}

\begin{tabular}{p{67mm} p{20mm} p{20mm} p{20mm} p{20mm} p{20mm}}
& Listening & Reading & Interaction & Production & \\
\end{tabular}

\vspace{3mm}
%lvl should be in this range A1 < A2 < B1 < B2 < C1 < C2
\cvlanguage{Italian}{Mother tongue}{
	\begin{tabular}{p{20mm} p{20mm} p{20mm} p{20mm} p{21mm}}
		C2 & C2 & C2 & C2 & C2
	\end{tabular}}
\cvlanguage{English}{Intermediate}{
	\begin{tabular}{p{20mm} p{20mm} p{20mm} p{20mm} p{21mm}}
		B2 & B2 & B2 & B2 & B2
	\end{tabular}}
\cvlanguage{French}{Basic}{
	\begin{tabular}{p{20mm} p{20mm} p{20mm} p{20mm} p{21mm}}
		B1 & B1 & B1 & B1 & B1
	\end{tabular}}
%Section
\section{Certificates}
\cventry{from 05/2013 to 05/2015}{CLAD - NI Certified LabVIEW Associate Developer}{National Instruments, Licenza 100-313-2906}{}{}{}
\cventry{05/2007}{Preliminary English Test(PET)}{University of Cambridge, Licenza 100/2031/7}{}{}{}
\cventry{05/2008}{B1 DELF - Diplôme D'Études En Langue Française}{}{}{}{}
\cventry{12/2020}{EF Level 11 - Upper Intermediate - CEFR Level B2 }{}{}{}{}
%Section





% Publications from a BibTeX file
%\nocite{*}
%\bibliographystyle{plain}
%\bibliography{publications}       % 'publications' is the name of a BibTeX file

\end{document}
